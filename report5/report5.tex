\documentclass[11pt,a4wide]{article}
\usepackage{verbatim}
\usepackage{listings}
\usepackage{graphicx}
\usepackage{a4wide}
\usepackage{color}
\usepackage{amsmath}
\usepackage{amssymb}
\usepackage[dvips]{epsfig}
\usepackage[T1]{fontenc}
\usepackage{cite} % [2,3,4] --> [2--4]
\usepackage{shadow}
\usepackage{hyperref}

\setcounter{tocdepth}{2}

\lstset{language=c++}
\lstset{alsolanguage=[90]Fortran}
\lstset{basicstyle=\small}
\lstset{backgroundcolor=\color{white}}
\lstset{frame=single}
\lstset{stringstyle=\ttfamily}
\lstset{keywordstyle=\color{red}\bfseries}
\lstset{commentstyle=\itshape\color{blue}}
\lstset{showspaces=false}
\lstset{showstringspaces=false}
\lstset{showtabs=false}
\lstset{breaklines}


%lager heftig forside:
\newcommand*{\titleAT}{\begingroup % Create the command for including the title page in the document
\newlength{\drop} % Command for generating a specific amount of whitespace
\drop=0.1\textheight % Define the command as 10% of the total text height

\rule{\textwidth}{1pt}\par % Thick horizontal line
\vspace{2pt}\vspace{-\baselineskip} % Whitespace between lines
\rule{\textwidth}{0.4pt}\par % Thin horizontal line

\vspace{0.5\drop} % Whitespace between the top lines and title
\centering % Center all text
\textcolor{black}{ % Red font color
{\Huge $N$-body simulation of an open galactic cluster}\\[0.75\baselineskip] % Title line 1
%{\Large Tema:}\\[0.75\baselineskip] % Title line 2
%{\Huge Lydmåling og hørselstesting} % Title line 3
} 

\vspace{0.25\drop} % Whitespace between the title and short horizontal line
\rule{0.3\textwidth}{0.4pt}\par % Short horizontal line under the title
\vspace{0.25\drop} % Whitespace between the thin horizontal line and the author name

{\Large \textsc{Project 5, FYS-3150\\[0.75\baselineskip] \normalsize{Ina K. B. Kullmann, candidate nr: 20}
}}\par % Author name

%\vfill % Whitespace between the author name and publisher text

\vspace{0.25\drop} % Whitespace between the title and short horizontal line
\rule{0.3\textwidth}{0.4pt}\par % Short horizontal line under the title
\vspace{0.25\drop} % Whitespace between the thin horizontal line and the author name

\begin{abstract}
The aim of this project is to numerically find the critical temperature for the two dimentional Ising model by using the metropolis algorithm. We will first test the implementation of the algorithm carefully, first by comparing with theoretical values calculated for a small system. Then we will see if the algorithm behaves as expected according to our physical intuition for a larger system.

When we have found a estimate for the critical temperature we will compare it to Lars Onsagers analytical result.

\end{abstract}
\vspace*{0.25\drop} % Whitespace under the publisher text

\begin{center}
{ \scriptsize \noindent All source codes can be found at: \texttt{https://github.com/inakbk/Project\_5}. }
\end{center}

\rule{\textwidth}{0.4pt}\par % Thin horizontal line
\vspace{2pt}\vspace{-\baselineskip} % Whitespace between lines
\rule{\textwidth}{1pt}\par % Thick horizontal line

\endgroup}
%kode slutt for heftig forside


\begin{document}
%\maketitle
\titleAT % This command includes the title page


\newpage
\tableofcontents
\newpage

\section{Introduction}

The goal in this project is to develop a code that can perform simulations of an open cluster using Newtonian gravity. First, however we will compare the stability of two different methods. This is because when we are looking at a system with a large number of particles, we are more interested in the statistical properties of the system than in the individual motion of each of the particles. This means that the stability of the solution method is more important than its short term accuracy.  This project is inspired by an article by Joyce {\em et al.}, see Ref.~\cite{joyce2010} below.

In the first part of this project we will explore the stability of two well-tested numerical methods for solving differential equations. The algorithms to test and implement are the fourth-order Runge-Kutta method and the Velocity-Verlet method. 

\section{Theory}
The Newtonian two-body problem in three dimensions

dimentionless variables

\section{Numerical methods}
Runge-Kutta method and the Velocity-Verlet method discussed in the lecture notes.
  
  
\rule{0.3\textwidth}{0.4pt}\par % Short horizontal line under the title  

\begin{enumerate}
\item[a)] Implement the Newtonian two-body (you can choose masses and dimensionalities as
  you wish) problem in three dimensions using the fourth order
  Runge-Kutta method and the Velocity-Verlet method discussed in
  the lecture notes.

 Compare the
stability of the two different methods. How do they work for large
time steps? How do they work for very long times? Compare also the
time used to advance one timestep for the two different
methods. Comment your results. Which algorithm would you use for
simulating systems that require long  times?

\end{enumerate}





We will now try to build a simple model of an open
cluster, see Ref.~\cite{openclusterref}. An
open cluster is a group of up to a few thousand gravitationally bound
stars created from the collapse of a molecular cloud. This collapse
leads to a flurry of star formation. Open clusters are usually found
in the arms of spiral galaxies, or in irregular galaxies. Since stars
in an open cluster have roughly the same age, and are made from the
same material, they are interesting in the study of stellar evolution,
since many of the variable parameters we have when comparing two stars
are kept constant.

Once open clusters are formed they gradually dissipate as members get
ejected from the cluster due to random collisions, this means that
open clusters generally last only a few hundred million years. In
figure \ref{HR}, we see the Hertzsprung-Russell diagrams for two open
clusters.

\begin{figure}[!h]
\centering
\includegraphics[width=10cm]{FigAstro/open_cluster_hr_diagram_ages.png}
\caption{Hertzsprung-Russell diagrams for two open clusters, M67 and
  NGC 188. We see that most of the stars are on the main sequence. In
  the older cluster, NGC 188, we see that the heaviest stars are just
  now leaving the main sequence, while the younger cluster, M67, is
  following closely after.}
\label{HR}
\end{figure}

We will look at a simple model for how an open cluster is made from
the gravitational collapse and interaction among a large number of
stars. We want to study this collapse, and the statistical properties
of the collapsed system.

One particle in our model represents one or a few stars, and we will
work first with a few hundred particles. We will simulate what is called a
``cold collapse'', this means that we start the particles with little
or no initial velocity.
\begin{enumerate}
\item[b)] Extend your code to an arbitrary number of particles, $N$,
  starting with a uniform (random) distribution within a sphere of a
  given radius $R_0$. Start the particles at rest, with masses
  randomly distributed by a Gaussian distribution around ten solar
  masses with a standard deviation of one solar mass. Use solar masses
  and light years as units of mass and length and make your equations dimensionless.
The function {\em GaussPDF} included with this project can be used to generate 
random numbers which follow a Gaussian (or normal) distribution.  The function for calculating these random numbers can be found at the webpage of the course together with the project files.

How large time steps are required given $R_0 = 20 ly$
(light years), and a $N = 100$? Do we have any units of time that fit
this timescale?   
In the limit where $N \rightarrow \infty$, keeping $\rho_0$ constant,
we get a continuous fluid. In this case the system collapses into a
singularity at a finite time $\tau_{crunch} = \sqrt{\frac{3\pi}{32G\rho_0}}$. 
(For the especially interested (Not required!): Can
you derive this result? Hint: recall the Friedman equations \cite{friedmaneqs}).

Why do we not observe this singularity in our model? Use
$\tau_{crunch}$ as the unit of time, and find $G$ in these units ($G$
will become a function of the number of particles $N$, and the average
mass of the particles, $\mu$).

You should run these calculations with both the fourth-order Runge-Kutta algorithm
and the Velocity-Verlet method. Which method would you prefer? Give a critical discussion.

For the remaining exercises, you should use only one of the above methods. 

\item[c)] Run the system for a few $\tau_{crunch}$. Save the positions
  of the particles at different times to file. Does the system
  reach an equilibrium? How long time does this take ?

\item[d)] Make a function that calculates the kinetic and potential
  energy of the system. Is the energy conserved? Some of the
  particles are ejected from the system, how can we identify these
  particles from the energies we have calculated? How much of the
  energy of the system is taken away by particle ejection? How does
  this change with different values of $N$? Are there still particles
  being ejected after the system reaches equilibrium?


\item[e)] We will now introduce a smoothing function to take care of
  the numerical instability that arises when two of the particles come
  very close. There are a lot of ways of inserting such a smoothing, but
  we will just look at a very simple one. We will modify the Newtonian
  force law to make it finite at short ranges
\[ F_{mod} = -\frac{GM_1M_2}{r^2 + \epsilon^2}.\]

The parameter $\epsilon$ is a ``small'' real constant. What should the value of this parameter be? Try out different values (which one gives you the best energy conservation?). Can we justify this correction to the pure Newtonian force by noting that our particles do not represent actual point particles
but rather mass distributions of some finite extent? Does the addition
of this correction change any of the results from part e) ?


\item[f)] Now we will look at the particles that are bound (not
  ejected). What is the distribution of potential and kinetic energy?

The virial theorem says that for a bound gravitational system in
equilibrium we have
\[
2\langle K\rangle = -\langle V \rangle,
\] 
where $\langle K\rangle$
is the average (over time) kinetic energy of the system and $ \langle V
\rangle$ is the average potential energy. 

By the ergodic hypothesis we can take an ensamble average (average over a large system) instead of the time average. 

Are your results consistent with the virial theorem? 

\item[h)]  Try to plot the radial density of the particles (the particle density as a function of radius) in the
  equilibrium state. How would you extract such an information from your calculations? (Hint: make a histogram for the radial
particle density) What is the average distance? What is the
  standard deviation? Plot the radial distribution of particles.

Run the code for different number of initial particles, keeping the
total mass constant. \footnote{The interpretation of one particle as one or a few stars will not be useful any more when you increase the number of particles beyond a few thousand, however the analysis of gravitationally bound systems of many particles have much broader applications than cold collapse of open clusters, so the results are still highly relevant.} What is the average distance from the centre of the cluser as a function of $N$?

The radial distribution of particles in this kind of cold collapse can often be fit very well with the simple expression 
\[ 
n(r) = \frac{n_0}{\left(1 +\left(\frac{r}{r_0}\right)^4\right)}.
\]
Try to fit your data to this curve, what is the value $n_0$ and $r_0$? Can you find how these values depend on N?

How many particles can you simulate?

Compare your results with those found in Ref.~\cite{joyce2010}. 

If you want, you can also compare your results to the well-known Navarro-Frenk-White profile
\[ 
\rho(r) = \frac{\rho_0}{\frac{r}{r_0}\left(1 +\frac{r}{r_0}\right)^2}.
\]
Does this fit better?

\end{enumerate}
\begin{thebibliography}{100} 
\bibitem{joyce2010} M.~Joyce, B.~Marcos, and F.~Sylos Labini, Cold uniform spherical collapse revisited, AIP Conf. Proc. 1241, 955 (2010); \url{http://dx.doi.org/10.1063/1.3462740} and arXiv1011.0614 (2011), 
\url{http://arxiv.org/abs/1011.0614}.
\bibitem{openclusterref} P.~J.~E. Peebles, \emph{{The Large-Scale Structure of the Universe}}, Princeton
  University Press, 1980. See also C.~Payne-Gaposchkin,  {\em Stars and clusters}, (Cambridge, Harvard University Press, 1979), or just \url{https://en.wikipedia.org/wiki/Open_cluster}. 
\bibitem{friedmaneqs} A.~Friedman,  {\em On the Curvature of Space},  General Relativity and Gravitation {\bf 31},  1991 (1999), or just \url{https://en.wikipedia.org/wiki/Friedmann_equations}.
\end{thebibliography} 
\end{document}







