\documentclass[11pt,a4wide]{article}
\usepackage{verbatim}
\usepackage{listings}
\usepackage{graphicx}
\usepackage{a4wide}
\usepackage{color}
\usepackage{amsmath}
\usepackage{amssymb}
\usepackage[dvips]{epsfig}
\usepackage[T1]{fontenc}
\usepackage{cite} % [2,3,4] --> [2--4]
\usepackage{shadow}
\usepackage{hyperref}

\setcounter{tocdepth}{2}

\lstset{language=c++}
\lstset{alsolanguage=[90]Fortran}
\lstset{basicstyle=\small}
\lstset{backgroundcolor=\color{white}}
\lstset{frame=single}
\lstset{stringstyle=\ttfamily}
\lstset{keywordstyle=\color{red}\bfseries}
\lstset{commentstyle=\itshape\color{blue}}
\lstset{showspaces=false}
\lstset{showstringspaces=false}
\lstset{showtabs=false}
\lstset{breaklines}


%lager heftig forside:
\newcommand*{\titleAT}{\begingroup % Create the command for including the title page in the document
\newlength{\drop} % Command for generating a specific amount of whitespace
\drop=0.1\textheight % Define the command as 10% of the total text height

\rule{\textwidth}{1pt}\par % Thick horizontal line
\vspace{2pt}\vspace{-\baselineskip} % Whitespace between lines
\rule{\textwidth}{0.4pt}\par % Thin horizontal line

\vspace{0.5\drop} % Whitespace between the top lines and title
\centering % Center all text
\textcolor{black}{ % Red font color
{\Huge Atomic modeling of argon}\\[0.75\baselineskip] % Title line 1
%{\Large Tema:}\\[0.75\baselineskip] % Title line 2
%{\Huge Lydmåling og hørselstesting} % Title line 3
} 

\vspace{0.25\drop} % Whitespace between the title and short horizontal line
\rule{0.3\textwidth}{0.4pt}\par % Short horizontal line under the title
\vspace{0.25\drop} % Whitespace between the thin horizontal line and the author name

{\Large \textsc{Project 5, FYS-3150\\[0.75\baselineskip] \normalsize{Ina K. B. Kullmann, candidate nr: 20}
}}\par % Author name

%\vfill % Whitespace between the author name and publisher text

\vspace{0.25\drop} % Whitespace between the title and short horizontal line
\rule{0.3\textwidth}{0.4pt}\par % Short horizontal line under the title
\vspace{0.25\drop} % Whitespace between the thin horizontal line and the author name

\begin{abstract}
The aim of this project is to numerically find the critical temperature for the two dimentional Ising model by using the metropolis algorithm. We will first test the implementation of the algorithm carefully, first by comparing with theoretical values calculated for a small system. Then we will see if the algorithm behaves as expected according to our physical intuition for a larger system.

When we have found a estimate for the critical temperature we will compare it to Lars Onsagers analytical result.

\end{abstract}
\vspace*{0.25\drop} % Whitespace under the publisher text

\begin{center}
{ \scriptsize \noindent All source codes can be found at: \texttt{https://github.com/inakbk/molecular-dynamics-fys3150}. }
\end{center}

\rule{\textwidth}{0.4pt}\par % Thin horizontal line
\vspace{2pt}\vspace{-\baselineskip} % Whitespace between lines
\rule{\textwidth}{1pt}\par % Thick horizontal line

\endgroup}
%kode slutt for heftig forside


\begin{document}
%\maketitle
\titleAT % This command includes the title page


\newpage
\tableofcontents
\newpage

\section{Introduction}

Molecular dynamics (MD) is a computer simulation method used to study atoms and molecule structure and movement. In a MD simulation the atoms or molecules are allowed to interact trough a force given by a potential for a given time. This makes it possible to study the systems development over time. 

MD is a a type of N-body simulation since the simulation often consists of a large number of atoms or molecules. It is therefore possible to use MD to study stastistical properties of a large system consisting of N such atoms or molecules. For systems that obey the ergodic hypothesis the evolution of a single molecular dynamics simulation may be used to determine macroscopic thermodynamic properties of the system. This is because the time averages of an ergodic system correspond to microcanonical ensemble averages\footnote{\texttt{https://en.wikipedia.org/wiki/Molecular\_dynamics} 3.dec 11:25}.

Often the main motivation to use Molecular dynamics is that it is not possible to determine properties of the system analytically because of the large number of particles. The main limitation for the numerical simulation is the computer recources available, but also cumulative errors in the numerical integration. The first is solved by applying periodic boundary conditions while the latter is solved by proper selection of algorithms and parameters. In this paper we will have a look at two numerical integration methods; the Euler-Cromer method and the Velocity Verlet integrator. 


In this paper we will study the properties of a large system consisting of Argon atoms. And compare with experimental data(?). We will have a constant number of particles, a constant volume and a more or less constant engergy (depending on the integrator). We are more interested in the statistical properties of the system than in the individual motion of each of the particles. We want to sample microstates from the microcanonical ensemble (NVE).
(?se over avsnittet over?)


The applications of MD is many ranging from chemical physics, materials science and the modelling of biomolecules.
What areas of physics can it be used in? Chemistry and biology?(fra oppg)
see 'Areas of application and limitations' at https://en.wikipedia.org/wiki/Molecular\_dynamics and google




\section{Theory}


\section{Numerical methods}
Euler-Cromer method and the Velocity-Verlet method (discussed in the lecture notes.)
  

\rule{0.3\textwidth}{0.4pt}\par % Short horizontal line under the title  

\end{document}







